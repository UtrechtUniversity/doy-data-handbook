% Options for packages loaded elsewhere
\PassOptionsToPackage{unicode}{hyperref}
\PassOptionsToPackage{hyphens}{url}
\PassOptionsToPackage{dvipsnames,svgnames,x11names}{xcolor}
%
\documentclass[
  letterpaper,
  DIV=11,
  numbers=noendperiod]{scrreprt}

\usepackage{amsmath,amssymb}
\usepackage{iftex}
\ifPDFTeX
  \usepackage[T1]{fontenc}
  \usepackage[utf8]{inputenc}
  \usepackage{textcomp} % provide euro and other symbols
\else % if luatex or xetex
  \usepackage{unicode-math}
  \defaultfontfeatures{Scale=MatchLowercase}
  \defaultfontfeatures[\rmfamily]{Ligatures=TeX,Scale=1}
\fi
\usepackage{lmodern}
\ifPDFTeX\else  
    % xetex/luatex font selection
\fi
% Use upquote if available, for straight quotes in verbatim environments
\IfFileExists{upquote.sty}{\usepackage{upquote}}{}
\IfFileExists{microtype.sty}{% use microtype if available
  \usepackage[]{microtype}
  \UseMicrotypeSet[protrusion]{basicmath} % disable protrusion for tt fonts
}{}
\makeatletter
\@ifundefined{KOMAClassName}{% if non-KOMA class
  \IfFileExists{parskip.sty}{%
    \usepackage{parskip}
  }{% else
    \setlength{\parindent}{0pt}
    \setlength{\parskip}{6pt plus 2pt minus 1pt}}
}{% if KOMA class
  \KOMAoptions{parskip=half}}
\makeatother
\usepackage{xcolor}
\setlength{\emergencystretch}{3em} % prevent overfull lines
\setcounter{secnumdepth}{5}
% Make \paragraph and \subparagraph free-standing
\makeatletter
\ifx\paragraph\undefined\else
  \let\oldparagraph\paragraph
  \renewcommand{\paragraph}{
    \@ifstar
      \xxxParagraphStar
      \xxxParagraphNoStar
  }
  \newcommand{\xxxParagraphStar}[1]{\oldparagraph*{#1}\mbox{}}
  \newcommand{\xxxParagraphNoStar}[1]{\oldparagraph{#1}\mbox{}}
\fi
\ifx\subparagraph\undefined\else
  \let\oldsubparagraph\subparagraph
  \renewcommand{\subparagraph}{
    \@ifstar
      \xxxSubParagraphStar
      \xxxSubParagraphNoStar
  }
  \newcommand{\xxxSubParagraphStar}[1]{\oldsubparagraph*{#1}\mbox{}}
  \newcommand{\xxxSubParagraphNoStar}[1]{\oldsubparagraph{#1}\mbox{}}
\fi
\makeatother


\providecommand{\tightlist}{%
  \setlength{\itemsep}{0pt}\setlength{\parskip}{0pt}}\usepackage{longtable,booktabs,array}
\usepackage{calc} % for calculating minipage widths
% Correct order of tables after \paragraph or \subparagraph
\usepackage{etoolbox}
\makeatletter
\patchcmd\longtable{\par}{\if@noskipsec\mbox{}\fi\par}{}{}
\makeatother
% Allow footnotes in longtable head/foot
\IfFileExists{footnotehyper.sty}{\usepackage{footnotehyper}}{\usepackage{footnote}}
\makesavenoteenv{longtable}
\usepackage{graphicx}
\makeatletter
\newsavebox\pandoc@box
\newcommand*\pandocbounded[1]{% scales image to fit in text height/width
  \sbox\pandoc@box{#1}%
  \Gscale@div\@tempa{\textheight}{\dimexpr\ht\pandoc@box+\dp\pandoc@box\relax}%
  \Gscale@div\@tempb{\linewidth}{\wd\pandoc@box}%
  \ifdim\@tempb\p@<\@tempa\p@\let\@tempa\@tempb\fi% select the smaller of both
  \ifdim\@tempa\p@<\p@\scalebox{\@tempa}{\usebox\pandoc@box}%
  \else\usebox{\pandoc@box}%
  \fi%
}
% Set default figure placement to htbp
\def\fps@figure{htbp}
\makeatother
% definitions for citeproc citations
\NewDocumentCommand\citeproctext{}{}
\NewDocumentCommand\citeproc{mm}{%
  \begingroup\def\citeproctext{#2}\cite{#1}\endgroup}
\makeatletter
 % allow citations to break across lines
 \let\@cite@ofmt\@firstofone
 % avoid brackets around text for \cite:
 \def\@biblabel#1{}
 \def\@cite#1#2{{#1\if@tempswa , #2\fi}}
\makeatother
\newlength{\cslhangindent}
\setlength{\cslhangindent}{1.5em}
\newlength{\csllabelwidth}
\setlength{\csllabelwidth}{3em}
\newenvironment{CSLReferences}[2] % #1 hanging-indent, #2 entry-spacing
 {\begin{list}{}{%
  \setlength{\itemindent}{0pt}
  \setlength{\leftmargin}{0pt}
  \setlength{\parsep}{0pt}
  % turn on hanging indent if param 1 is 1
  \ifodd #1
   \setlength{\leftmargin}{\cslhangindent}
   \setlength{\itemindent}{-1\cslhangindent}
  \fi
  % set entry spacing
  \setlength{\itemsep}{#2\baselineskip}}}
 {\end{list}}
\usepackage{calc}
\newcommand{\CSLBlock}[1]{\hfill\break\parbox[t]{\linewidth}{\strut\ignorespaces#1\strut}}
\newcommand{\CSLLeftMargin}[1]{\parbox[t]{\csllabelwidth}{\strut#1\strut}}
\newcommand{\CSLRightInline}[1]{\parbox[t]{\linewidth - \csllabelwidth}{\strut#1\strut}}
\newcommand{\CSLIndent}[1]{\hspace{\cslhangindent}#1}

\KOMAoption{captions}{tableheading}
\makeatletter
\@ifpackageloaded{bookmark}{}{\usepackage{bookmark}}
\makeatother
\makeatletter
\@ifpackageloaded{caption}{}{\usepackage{caption}}
\AtBeginDocument{%
\ifdefined\contentsname
  \renewcommand*\contentsname{Table of contents}
\else
  \newcommand\contentsname{Table of contents}
\fi
\ifdefined\listfigurename
  \renewcommand*\listfigurename{List of Figures}
\else
  \newcommand\listfigurename{List of Figures}
\fi
\ifdefined\listtablename
  \renewcommand*\listtablename{List of Tables}
\else
  \newcommand\listtablename{List of Tables}
\fi
\ifdefined\figurename
  \renewcommand*\figurename{Figure}
\else
  \newcommand\figurename{Figure}
\fi
\ifdefined\tablename
  \renewcommand*\tablename{Table}
\else
  \newcommand\tablename{Table}
\fi
}
\@ifpackageloaded{float}{}{\usepackage{float}}
\floatstyle{ruled}
\@ifundefined{c@chapter}{\newfloat{codelisting}{h}{lop}}{\newfloat{codelisting}{h}{lop}[chapter]}
\floatname{codelisting}{Listing}
\newcommand*\listoflistings{\listof{codelisting}{List of Listings}}
\makeatother
\makeatletter
\makeatother
\makeatletter
\@ifpackageloaded{caption}{}{\usepackage{caption}}
\@ifpackageloaded{subcaption}{}{\usepackage{subcaption}}
\makeatother

\usepackage{bookmark}

\IfFileExists{xurl.sty}{\usepackage{xurl}}{} % add URL line breaks if available
\urlstyle{same} % disable monospaced font for URLs
\hypersetup{
  pdftitle={Dynamics of Youth},
  pdfauthor={Neha Moopen},
  colorlinks=true,
  linkcolor={blue},
  filecolor={Maroon},
  citecolor={Blue},
  urlcolor={Blue},
  pdfcreator={LaTeX via pandoc}}


\title{Dynamics of Youth}
\usepackage{etoolbox}
\makeatletter
\providecommand{\subtitle}[1]{% add subtitle to \maketitle
  \apptocmd{\@title}{\par {\large #1 \par}}{}{}
}
\makeatother
\subtitle{DATA HANDBOOK}
\author{Neha Moopen}
\date{2025-10-15}

\begin{document}
\maketitle

\renewcommand*\contentsname{Table of contents}
{
\hypersetup{linkcolor=}
\setcounter{tocdepth}{2}
\tableofcontents
}

\bookmarksetup{startatroot}

\chapter*{Welcome!}\label{welcome}
\addcontentsline{toc}{chapter}{Welcome!}

\markboth{Welcome!}{Welcome!}

\begin{figure}[H]

{\centering \pandocbounded{\includegraphics[keepaspectratio]{images/fair-1x4.png}}

}

\caption{This illustration is created by Scriberia with The Turing Way
community. Used under a CC-BY 4.0 licence. DOI: 10.5281/zenodo.3332807}

\end{figure}%

\bookmarksetup{startatroot}

\chapter*{Definitions}\label{definitions}
\addcontentsline{toc}{chapter}{Definitions}

\markboth{Definitions}{Definitions}

Before diving into the Handbook, it would be good to get familiarized
with some data-related terms that are oftentimes misunderstood or used
interchangeably.

\section*{Research Data Management}\label{research-data-management}
\addcontentsline{toc}{section}{Research Data Management}

\markright{Research Data Management}

Research Data Management (RDM) refers to the active organization and
maintenance of data created during a research project. It is an ongoing
activity throughout the data lifecycle, from initial planning to
suitable archiving of the data at the project's completion.

\section*{FAIR Data}\label{fair-data}
\addcontentsline{toc}{section}{FAIR Data}

\markright{FAIR Data}

The FAIR Data Principles are a set of guiding principles to improve
scientific data management and stewardship
(\href{https://doi.org/10.1038/sdata.2016.18}{Wilkinson et al., 2016})

\begin{itemize}
\tightlist
\item
  FINDABILITY makes it possible for others to discover your data
  (metadata, Persistent Identifiers, etc.).
\item
  ACCESSIBILITY makes it possible for humans and machines to gain access
  to your data, under specific conditions or restrictions where
  appropriate.
\item
  INTEROPERABILITY ensures data and metadata conform to recognized
  formats and standards which allows them to be combined and exchanged.
\item
  REUSABILITY requires lots of documentation, which is needed to support
  data and interpretation and reuse.
\end{itemize}

\section*{Open Data}\label{open-data}
\addcontentsline{toc}{section}{Open Data}

\markright{Open Data}

Open Data is data that can be freely used, re-used, and redistributed by
anyone - subject only, at most, to the requirement to attribute and
share-alike
(\href{https://opendatahandbook.org/guide/en/what-is-open-data/}{Open
Data Handbook}).

Note that your data does not have to be `open' to be FAIR! Make your
data\ldots{} `as open as possible, as closed as necessary'
(\href{https://ec.europa.eu/research/participants/docs/h2020-funding-guide/cross-cutting-issues/open-access-data-management/data-management_en.htm}{European
Commission}).

\section*{Summary}\label{summary}
\addcontentsline{toc}{section}{Summary}

\markright{Summary}

In short,

\begin{itemize}
\tightlist
\item
  RDM = an activity/practice
\item
  FAIR = principles that guide RDM activities/practices
\item
  Open Data = data does not have to be `open' to be FAIR!
\end{itemize}

\bookmarksetup{startatroot}

\chapter*{Data Management Plans}\label{data-management-plans}
\addcontentsline{toc}{chapter}{Data Management Plans}

\markboth{Data Management Plans}{Data Management Plans}

\pandocbounded{\includegraphics[keepaspectratio]{images/data-management-plan.jpg}}

The Turing Way project illustration by Scriberia. Used under a CC-BY 4.0
licence. DOI: 10.5281/zenodo.3332807.

\section*{What Is A Data Management
Plan?}\label{what-is-a-data-management-plan}
\addcontentsline{toc}{section}{What Is A Data Management Plan?}

\markright{What Is A Data Management Plan?}

A Data Management Plan (DMP) is a formal document that describes your
data and outlines all aspects of managing your data - both during and
after your project.

Moreover, it is a \emph{living} document that can you can revise and
update as needed.

\section*{Why Should You Write A DMP?}\label{why-should-you-write-a-dmp}
\addcontentsline{toc}{section}{Why Should You Write A DMP?}

\markright{Why Should You Write A DMP?}

Writing a DMP provides an opportunity to reflect on your data,
particularly how you organize and manage it. It nudges you to think
about how to make your RDM more \emph{concrete} and \emph{actionable}.
This creates efficiency and more value for your data.

\section*{When Should You Write A
DMP?}\label{when-should-you-write-a-dmp}
\addcontentsline{toc}{section}{When Should You Write A DMP?}

\markright{When Should You Write A DMP?}

Working on a DMP at the start of your project will ensure that you are
better informed of best practices in RDM and prepared to implement them.
That being said, you can also write a DMP can during the project or when
it's completed.

\section*{DMPonline \& DMP Templates}\label{dmponline-dmp-templates}
\addcontentsline{toc}{section}{DMPonline \& DMP Templates}

\markright{DMPonline \& DMP Templates}

DMPonline is a tool that helps you create and maintain DMPs. With
DMPonline, you can:

\begin{itemize}
\tightlist
\item
  register and sign in with your institutional credentials,
\item
  write and collaborate on (multiple) DMPs,
\item
  share DMPs or switch their visibility between private and public,
\item
  request feedback from RDM Support,
\item
  download DMPs in various formats.
\end{itemize}

DMPonline offers DMP templates from various institutions and funders,
including:

\begin{itemize}
\tightlist
\item
  Utrecht University
\item
  UMC Utrecht
\item
  \href{https://dmponline.dcc.ac.uk/template_export/1753695087.pdf}{NWO}
\item
  \href{https://dmponline.dcc.ac.uk/template_export/1461074155.pdf}{ZonMw}
\item
  \href{https://dmponline.dcc.ac.uk/template_export/2088403152.pdf}{ERC}
\item
  \href{https://dmponline.dcc.ac.uk/template_export/1612436782.pdf}{Horizon
  2020}
\item
  \href{https://dmponline.dcc.ac.uk/template_export/5992485.pdf}{Horizon
  Europe}
\end{itemize}

These templates also contain example answers and guidance.

\pandocbounded{\includegraphics[keepaspectratio]{images/uu-dmp-template.JPG}}

\section*{Tips}\label{tips}
\addcontentsline{toc}{section}{Tips}

\markright{Tips}

!!! note ``Tips''

\begin{verbatim}
- Contact your DoY data manager! They can (co)write your DMP and/or review it.
- If the DoY data manager is unavailable, you can still request feedback from RDM Support.
\end{verbatim}

\section*{Resources}\label{resources}
\addcontentsline{toc}{section}{Resources}

\markright{Resources}

\begin{itemize}
\tightlist
\item
  \href{https://www.uu.nl/en/research/research-data-management/tools-services/tool-to-create-your-dmp-online}{Create
  your DMP online}
\item
  \href{https://www.uu.nl/en/research/research-data-management/guides/data-management-planning}{Data
  management planning}
\item
  \href{https://www.uu.nl/en/research/research-data-management/training-workshops/online-training-learn-to-write-your-dmp}{Learn
  to write your DMP (online training)}
\end{itemize}

\section*{References}\label{references}
\addcontentsline{toc}{section}{References}

\markright{References}

\begin{enumerate}
\def\labelenumi{\arabic{enumi}.}
\item
  \url{https://www.uu.nl/en/research/research-data-management/guides/data-management-planning}
\item
  \url{https://www.kuleuven.be/rdm/en/faq/faq-dmp}
\item
  \url{https://rdm.uva.nl/en/planning/data-management-plan/data-management-plan.html}
\item
  \url{https://www.uu.nl/en/research/research-data-management/tools-services/tool-to-create-your-dmp-online.html}
\end{enumerate}

\bookmarksetup{startatroot}

\chapter*{Data Flow Diagrams}\label{data-flow-diagrams}
\addcontentsline{toc}{chapter}{Data Flow Diagrams}

\markboth{Data Flow Diagrams}{Data Flow Diagrams}

A data flow diagram (DPF) is a visual representation of the flow of data
through a process or system. It provides an overview of incoming and
outgoing data, as well as the processing and tools involved.

A DFD is valuable because it provides an outline of your data processes.
It allows you to see how these processes interact and identify
opportunities for improvement.

DFDs can be as simple as hand-drawn flowcharts on an A4 sheet of paper
to elaborate flowcharts with different symbols and markers. It is
recommended that you sketch a DPD while working on your DMP. It can also
provide the basis for developing your data pipeline.

\section*{Examples}\label{examples}
\addcontentsline{toc}{section}{Examples}

\markright{Examples}

\subsection*{YOUth}\label{youth}
\addcontentsline{toc}{subsection}{YOUth}

\pandocbounded{\includegraphics[keepaspectratio]{images/dfd-youth.jpg}}

\subsection*{PFIC}\label{pfic}
\addcontentsline{toc}{subsection}{PFIC}

\pandocbounded{\includegraphics[keepaspectratio]{images/dfd-pfic.png}}

\subsection*{Hear, Hear}\label{hear-hear}
\addcontentsline{toc}{subsection}{Hear, Hear}

\subsection*{Smart-Youth}\label{smart-youth}
\addcontentsline{toc}{subsection}{Smart-Youth}

\bookmarksetup{startatroot}

\chapter*{Naming Conventions}\label{naming-conventions}
\addcontentsline{toc}{chapter}{Naming Conventions}

\markboth{Naming Conventions}{Naming Conventions}

\pandocbounded{\includegraphics[keepaspectratio]{images/xkcd-file-naming.png}}

Documents - xkcd. Used under a CC BY-NC 2.5 license.

\section*{What Is A Naming
Convention?}\label{what-is-a-naming-convention}
\addcontentsline{toc}{section}{What Is A Naming Convention?}

\markright{What Is A Naming Convention?}

A naming convention is a set of rules for naming things. You can apply
it to things like folders, files, and variables.

\section*{Why Should I Apply A Naming
Convention?}\label{why-should-i-apply-a-naming-convention}
\addcontentsline{toc}{section}{Why Should I Apply A Naming Convention?}

\markright{Why Should I Apply A Naming Convention?}

Names that are informative and useful for machines and humans are a step
toward efficient data management and reproducible research. The more
consistent and meaningful the name, the easier it will be to locate and
identify things, understand what they contain, and (re)use them.

\section*{When Should I Apply A Naming
Convention?}\label{when-should-i-apply-a-naming-convention}
\addcontentsline{toc}{section}{When Should I Apply A Naming Convention?}

\markright{When Should I Apply A Naming Convention?}

Aim to select and implement a naming convention at the beginning of a
project. If you want to retroactively apply a naming convention, there
are several tools for bulk renaming.

The entire research team should agree on and adopt a naming convention.
Document the choice of naming convention in the DMP, so others can refer
to and grasp it quickly.

\section*{Popular Naming Conventions}\label{popular-naming-conventions}
\addcontentsline{toc}{section}{Popular Naming Conventions}

\markright{Popular Naming Conventions}

Instead of developing a naming convention from scratch, you can start
with one that is already being used in programming and software
development communities:

\begin{longtable}[]{@{}
  >{\raggedright\arraybackslash}p{(\linewidth - 4\tabcolsep) * \real{0.3636}}
  >{\raggedright\arraybackslash}p{(\linewidth - 4\tabcolsep) * \real{0.3864}}
  >{\raggedright\arraybackslash}p{(\linewidth - 4\tabcolsep) * \real{0.2500}}@{}}
\toprule\noalign{}
\begin{minipage}[b]{\linewidth}\raggedright
Naming Covention
\end{minipage} & \begin{minipage}[b]{\linewidth}\raggedright
Example
\end{minipage} & \begin{minipage}[b]{\linewidth}\raggedright
Description
\end{minipage} \\
\midrule\noalign{}
\endhead
\bottomrule\noalign{}
\endlastfoot
original name & \texttt{an\ awesome\ name} & N/A \\
snake\_case & \texttt{an\_awesome\_name} & All words are lowercase and
separated by an underscore ( \texttt{\_} ) \\
kebab-case & \texttt{an-awesome-name} & All words are lowercase and
separated by a hyphen ( \texttt{-} ) \\
PascalCase & \texttt{AnAwesomeName} & All words are capitalized. Spaces
are not used. \\
camelCase & \texttt{anAwesomeName} & The first word is lowercase, the
remaining words are capitalized. Spaces are not used. \\
\end{longtable}

\section*{Human-Readable Names}\label{human-readable-names}
\addcontentsline{toc}{section}{Human-Readable Names}

\markright{Human-Readable Names}

You can tailor naming conventions like \texttt{snake\_case} and
\texttt{PascalCase} to suit your project and workflow. Determine what
information is relevant (or not) to create meaningful names and how you
can string this information together. Don't forget to document this in
your DMP!

!!! note ``Elements for Human-Readable Names''

\begin{verbatim}
Names should be =<25 characters long and can include:

- Date of creation/update (`YYYY-MM-DD` or `YYYYMMDD`)
- Description of content, like type of data
- Initials of creator/reviewer
- Project number or acronym
- Location/coordinates
- Version number (like `v2` or v2.2`)
\end{verbatim}

\section*{Machine-Readable Names}\label{machine-readable-names}
\addcontentsline{toc}{section}{Machine-Readable Names}

\markright{Machine-Readable Names}

When names are machine-readable, they can be efficiently processed by
computers and software. This makes it easier to search for files and run
operations that involve programming like extracting information from
file names or working with regular expressions.

!!! note ``Avoid''

\begin{verbatim}
- Spaces
- Special characters like `$`, `@`, `%`, `#`, `&`, `*`, `!`, `/`, `\`
- Punction characters like `,`, `:`, `;`, `?`, `'`, `"`
- Accented characters
\end{verbatim}

\section*{A Note on Numbering, Dates,
Versioning}\label{a-note-on-numbering-dates-versioning}
\addcontentsline{toc}{section}{A Note on Numbering, Dates, Versioning}

\markright{A Note on Numbering, Dates, Versioning}

\begin{itemize}
\item
  Append numbers to the beginning of a name to enable sorting according
  to a logical structure. Use multiple digits like \texttt{01} or
  \texttt{001}.
\item
  Dates should follow the ISO 8601 standard which is either
  \texttt{YYYY-MM-DD} or \texttt{YYYYMMDD}. Append dates to the
  beginning of names to enable sorting in chronological order.
\item
  Specify versions using ordinal numbers (1,2,3) for major revisions and
  decimals for minor changes (1.1, 1.2, 2.1, 2.2). Alternatively, you
  can specify versions with multiple digits like v01 and v02.
\end{itemize}

\section*{Renaming files}\label{renaming-files}
\addcontentsline{toc}{section}{Renaming files}

\markright{Renaming files}

The following tools enable renaming in bulk:

\begin{itemize}
\tightlist
\item
  \href{https://www.bulkrenameutility.co.uk/}{Bulk Rename Utility}
  (Windows, free)
\item
  \href{https://renamer.com/}{Renamer} (MacOS, paid)
\item
  \href{https://mrrsoftware.com/namechanger/}{NameChanger}, (MacOS,
  free)
\item
  \href{https://gprename.sourceforge.net/}{GPRename} (Linux, free)
\end{itemize}

\section*{References}\label{references-1}
\addcontentsline{toc}{section}{References}

\markright{References}

\begin{enumerate}
\def\labelenumi{\arabic{enumi}.}
\tightlist
\item
  \url{https://en.wikipedia.org/wiki/Naming_convention}
\item
  \url{https://help.osf.io/article/146-file-naming}
\item
  \url{https://rdm.elixir-belgium.org/file_naming.html}
\item
  \url{https://khalilstemmler.com/blogs/camel-case-snake-case-pascal-case/}
\item
  \url{https://dev.to/chaseadamsio/most-common-programming-case-types-30h9}
\item
  \url{https://rdmkit.elixir-europe.org/data_organisation}
  \url{http://dataabinitio.com/?p=987}
\item
  \url{https://dmeg.cessda.eu/Data-Management-Expert-Guide/2.-Organise-Document/File-naming-and-folder-structure}
\item
  \url{https://annakrystalli.me/rrresearchACCE20/filenaming-view.html}
\end{enumerate}

\bookmarksetup{startatroot}

\chapter*{Data Pipelining}\label{data-pipelining}
\addcontentsline{toc}{chapter}{Data Pipelining}

\markboth{Data Pipelining}{Data Pipelining}

A data pipeline is a series of (automated) actions that ingests raw data
from various sources and moves the data to a destination for storage and
(eventual) analysis.

Benefits of a data pipeline include:

\begin{itemize}
\tightlist
\item
  Time saved by automating the boring stuff!
\item
  Reduced mistakes.
\item
  Tasks broken down into smaller steps.
\item
  Reproducibility!
\end{itemize}

\section*{When do I need a data
pipeline?}\label{when-do-i-need-a-data-pipeline}
\addcontentsline{toc}{section}{When do I need a data pipeline?}

\markright{When do I need a data pipeline?}

Here's a rule of thumb, just as an example:

If you have a task that needs to occur \textgreater= 3 times, you could
think about automating it.

If automation is not possible, think about how you can make the task as
efficient as possible.

\section*{How can I implement a data pipeline? Some examples for
inspiration}\label{how-can-i-implement-a-data-pipeline-some-examples-for-inspiration}
\addcontentsline{toc}{section}{How can I implement a data pipeline? Some
examples for inspiration}

\markright{How can I implement a data pipeline? Some examples for
inspiration}

\begin{itemize}
\item
  If you data collection tools have APIs, they can be leveraged to
  extract data.
\item
  For example, Qualtrics has the qualtRics R package \& pyQualtrics
  Python library which contain functions to automate exporting surveys.
\item
  If APIs are not available, you could use R/Python to automate the use
  of an internet browser using the RSelenium package / Selenium library.
  Imagine automating the clicks and typing of going to a specific
  website, logging in, clicking the download button.
\item
  You can use Windows Task Scheduler / cron / the taskscheduleR R
  package / cronR to schedule your scripts to run automatically, on a
  recurring basis as well (if needed).
\item
  You can also send emails with R \& Python! Consider if you've ever had
  to contact participants because you noticed something wrong with their
  incoming data. You could implement these data checks with a script and
  automatically draft and send emails (from a template) to those
  participants who were flagged as having issues with their data.
\end{itemize}

\section*{QualtRics R package}\label{qualtrics-r-package}
\addcontentsline{toc}{section}{QualtRics R package}

\markright{QualtRics R package}

\begin{verbatim}
library(readr)
library(qualtRics)

qualtrics_api_credentials(api_key = "YOUR-QUALTRICS-API-KEY", 
                          base_url = "YOUR-QUALTRICS-BASE-URL",
                          overwrite = TRUE,
                          install = TRUE)

readRenviron("~/.Renviron")

surveys <- all_surveys() 

survey_results <- fetch_survey(surveyID = surveys$id[2], # you can also replace surveys$id[2] with "<SUVREY-ID>" 
                                  verbose = TRUE)

write_csv(survey_results, paste0("path/to/folder/", format(Sys.time(), "%d-%m-%Y-%H.%M"), "_survey_results.csv"))
\end{verbatim}

\section*{taskscheduleR package}\label{taskscheduler-package}
\addcontentsline{toc}{section}{taskscheduleR package}

\markright{taskscheduleR package}

\begin{verbatim}
library(taskscheduleR)

scheduled_script <- "path/to/folder/myscript.R"

## run script once within 120 seconds

taskscheduler_create(taskname = "extract-data-once", rscript = scheduled_script,
                     schedule = "ONCE", starttime = format(Sys.time() + 120, "%H:%M"))

## Run every 5 minutes, starting from 10:40

taskscheduler_create(taskname = "extract-data-5min", rscript = scheduled_script,
                     schedule = "MINUTE", starttime = "10:40", modifier = 5)

## delete tasks

taskscheduler_delete("extract-data-once")
\end{verbatim}

\bookmarksetup{startatroot}

\chapter*{Metadata}\label{metadata}
\addcontentsline{toc}{chapter}{Metadata}

\markboth{Metadata}{Metadata}

Metadata is structured information that describes one or more aspects of
your research data. In other words, metadata = `data about data'.
Metadata is machine-readable and helps make your data findable and
citable.

Metadata exists at different levels:

\section*{Project-Level Metadata}\label{project-level-metadata}
\addcontentsline{toc}{section}{Project-Level Metadata}

\markright{Project-Level Metadata}

This type of metadata describes higher-order aspects of your dataset:
the ``who, what, where, when, how and why'' \ldots{} It provides context
for understanding why the data were collected and how they were used.

• Name of the project • Dataset title • Project description • Dataset
abstract • Principal investigator and collaborators • Contact
information • Dataset handle (DOI or URL) • Dataset citation • Data
publication date • Geographic description • Time period of data
collection • Subject/keywords • Project sponsor • Dataset usage rights

\section*{Data-Level Metadata}\label{data-level-metadata}
\addcontentsline{toc}{section}{Data-Level Metadata}

\markright{Data-Level Metadata}

• Data origin: experimental, observational, raw or derived, physical
collections, models, images, etc. • Data type: integer, Boolean,
character, floating point, etc. • Instrument(s) used • Data acquisition
details: sensor deployment methods, experimental design, sensor
calibration methods, etc. • File type: CSV, mat, xlsx, tiff, HDF,
NetCDF, etc. • Data processing methods, software used • Data processing
scripts or codes • Dataset parameter list, including ⚬ Variable names ⚬
Description of each variable ⚬ Units

This type of metadata is more granular and describes the data
(variables) and dataset in detail.

\bookmarksetup{startatroot}

\chapter*{Documentation}\label{documentation}
\addcontentsline{toc}{chapter}{Documentation}

\markboth{Documentation}{Documentation}

Documentation refers to contextual information pertaining to your
research data. It accompanies (structured) metadata and guides users to
understand and interpret your data and reuse it effectively.

Documentation is meant to be human-readable and it is a crucial aspect
of interoperability and reusability. Some examples include:

• Grant / Study Proposals • Study Protocol / Methodology • Data
Management Plan (DMP) • README files • Lab Notebooks • Legal / Policy /
Administrative Documents

\section*{Documentation Checklist}\label{documentation-checklist}
\addcontentsline{toc}{section}{Documentation Checklist}

\markright{Documentation Checklist}

Here is a starter checklist to make an inventory of your documentation:
\url{https://tinyurl.com/documentation-checklist}

\bookmarksetup{startatroot}

\chapter*{Codebooks}\label{codebooks}
\addcontentsline{toc}{chapter}{Codebooks}

\markboth{Codebooks}{Codebooks}

A codebook is an example of data-level metadata.

The purpose of a codebook or data dictionary is to explain what all the
variable names and values in your spreadsheet really mean.

Information to include in a codebook includes:

\begin{itemize}
\tightlist
\item
  Variable Names
\item
  Readable Variable Name
\item
  Measurement Units
\item
  Allowed Values
\item
  Definition Of The Variable
\item
  Synonyms For The Variable Name (Optional)
\item
  Description Of The Variable (Optional)
\item
  Other Resources
\end{itemize}

See: \url{https://help.osf.io/article/217-how-to-make-a-data-dictionary}

\section*{codebook R package}\label{codebook-r-package}
\addcontentsline{toc}{section}{codebook R package}

\markright{codebook R package}

\begin{verbatim}
library(qualtRics)
library(readr)
library(dplyr)
library(codebook)
library(writexl)

surveys <- all_surveys()

survey_results <- fetch_survey(surveyID = surveys$id[2], # you can also replace surveys$id[2] with "<SUVREY-ID>"
                               verbose = TRUE)

survey_results <- select(survey_results, -c(1:17))

# survey_questions() retrieves a data frame containing questions and question IDs for a survey;
survey_questions <- survey_questions(surveyID = surveys$id[2])
survey_questions <- select(survey_questions, -c(1, 4))
survey_questions <- slice(survey_questions, -1)
  
# generate codebook

codebook <- codebook_table(survey_results)

codebook <- rename(codebook, qname = name)

codebook <- full_join(survey_questions, codebook, by = "qname")

write_xlsx(codebook, "documentation/codebook-demo.xlsx")
\end{verbatim}

The \texttt{labelled} R package can also do something similar.

\bookmarksetup{startatroot}

\chapter*{Data Storage}\label{data-storage}
\addcontentsline{toc}{chapter}{Data Storage}

\markboth{Data Storage}{Data Storage}

When discussing storage, we are considering the location of `active'
data is under use and subject to change during the research project. The
related concepts of archiving and publishing refer to where the data
will be saved or deposited after the project is completed.

When storing data, consider the following - choose storage media that is
appropriate for the type of data you're working with - implement
reliable version control and backups - structure folders and organize
files clearly - follow a naming convention - use preferred and
sustainable file formats - secure data files

\section*{Data Storage Finder}\label{data-storage-finder}
\addcontentsline{toc}{section}{Data Storage Finder}

\markright{Data Storage Finder}

The \href{https://tools.uu.nl/storagefinder/}{Data Storage Finder} is a
tool provided by IT help you decide which storage solution would be most
suited to your needs.

\bookmarksetup{startatroot}

\chapter*{Data Archving}\label{data-archving}
\addcontentsline{toc}{chapter}{Data Archving}

\markboth{Data Archving}{Data Archving}

Data Archiving refers to the long-term preservation of research data. It
is typically done for verification purposes / to check \& maintain the
integrity of the original research.

There are varying policies on how long research data should be retained
for verification purposes, a typical policy is 10 years for the
preservation of raw data.

Archiving is not directly related to the FAIR principles, since the
latter is focused on sharing and reusing the data. Nonetheless, the
steps taken in archiving can provide a bsis for FAIRification, so the
effort is never wasted!

\bookmarksetup{startatroot}

\chapter*{Data Publication}\label{data-publication}
\addcontentsline{toc}{chapter}{Data Publication}

\markboth{Data Publication}{Data Publication}

When publishing (meta)data, you want to make it findable and reusable.
The data (and information about the data) can be used by others for
their own purposes. It's up to you to specify the terms and conditions
for access and reuse.

Note that your data need not be `open' to be FAIR! The data files
themselves can be placed under restricted access (or retained
internally) while the metadata and documentation are openly published.
Once any data sharing agreements are signed, the data files an be
transferred according to best practices.

When publishing (meta)data, you will receive a landing page for your
dataset and a DOI (persistent identifier) that makes it findable and
citeable. When you include your metadata and documentation, you improve
accessibility and reusability.

\section*{Examples}\label{examples-1}
\addcontentsline{toc}{section}{Examples}

\markright{Examples}

\begin{itemize}
\tightlist
\item
  Nijhof, Sanne; Putte, Elise van de; Hoefnagels, Johanna Wilhelmina,
  2021, ``PROactive Cohort Study'', https://doi.org/10.34894/FXUGHW,
  DataverseNL, V3
\item
  Isabelle van der Linden; Henk Schipper; Sanne Nijhof; Kors van der
  Ent, 2024, ``SMART-Youth: Data'', https://doi.org/10.34894/FCBXSI,
  DataverseNL, V1
\end{itemize}

\section*{Tools}\label{tools}
\addcontentsline{toc}{section}{Tools}

\markright{Tools}

You can use the UU Data Repository Finder and see which data repository
might be most suitable for publishing your project.

\bookmarksetup{startatroot}

\chapter*{Data Governance}\label{data-governance}
\addcontentsline{toc}{chapter}{Data Governance}

\markboth{Data Governance}{Data Governance}

When you're ready to start sharing your data, you can set up a detailed
Data Access Protocol (DAP) that outlines the data governance for
yourself, your research team, and potential re-users. This DAP will
ideally be public and findable in your chosen repository.

There are many topics within a DAP, it will require you (and/or the
project team to come together) to decide on what is relevant and best
for your data. This can include, for example, the terms \& conditions
for data reuse and the governance procedure in terms of responsibilities
and tasks of the team members.

See the PROactive Cohort Study's DAP here:
https://dataverse.nl/file.xhtml?fileId=141206\&version=3.0

Data Governance can be as simple and elaborate as you like, it all
depends on you and your project team.

Reflect on:

• What would you like to get out of sharing the data? For example,
citations/acknowledgments, co-authorship, collaboration? This should be
specified in the DAP so the end-user knows their obligations.

• What kind of time and effort can you and/or your team invest in the
data governance? For example, assessing incoming requests, preparing a
datafile for sharing, maintaining a data sharing logbook. Note: If there
is privacy-sensitive data involved, even the simplest DAPs have to take
some legal considerations into account!

\bookmarksetup{startatroot}

\chapter*{Data Sharing}\label{data-sharing}
\addcontentsline{toc}{chapter}{Data Sharing}

\markboth{Data Sharing}{Data Sharing}

When sharing (personal) data with collaborators outside the university,
there are a couple of important considerations:

\begin{itemize}
\tightlist
\item
  The participation letter and informed consent forms should have
  clearly informed participants about data sharing and reuse + they
  should agree to it.
\item
  A Data Protection Impact Assessment may have to be carried out, this
  will reveal to what extent it is safe to share data (or not) and how
  that can be put into practice (for example, pseudonymization
  techniques)
\item
  Any transfer of data outside the UU will require a Data Transfer
  Agreement in line with the GDPR, the complexity of the DTA will vary
  depending on the nature of the transfer (for example, transfer outside
  the EU).
\end{itemize}

\section*{Tools}\label{tools-1}
\addcontentsline{toc}{section}{Tools}

\markright{Tools}

\subsection*{SURFfilesender}\label{surffilesender}
\addcontentsline{toc}{subsection}{SURFfilesender}

SURFFileSender is a reliable tool to send data to another user. You can
send large files securely and the option for encryption makes it more
safe.

\subsection*{Virtual Research
Environments}\label{virtual-research-environments}
\addcontentsline{toc}{subsection}{Virtual Research Environments}

VREs, for example - AnDREa \& ResearchCloud, is a temporary computing
environment that is secure and contains the necessary tools and files
for users to carry out some research activities.

\bookmarksetup{startatroot}

\chapter*{FAIR Data Cheatsheet}\label{fair-data-cheatsheet}
\addcontentsline{toc}{chapter}{FAIR Data Cheatsheet}

\markboth{FAIR Data Cheatsheet}{FAIR Data Cheatsheet}

\bookmarksetup{startatroot}

\chapter*{References}\label{references-2}
\addcontentsline{toc}{chapter}{References}

\markboth{References}{References}

\phantomsection\label{refs}
\begin{CSLReferences}{0}{1}
\end{CSLReferences}

\part{APPENDIX}

\chapter*{Trainings}\label{trainings}
\addcontentsline{toc}{chapter}{Trainings}

\markboth{Trainings}{Trainings}

\section*{2024}\label{section}
\addcontentsline{toc}{section}{2024}

\markright{2024}

\subsection*{Managing Qualitative Data}\label{managing-qualitative-data}
\addcontentsline{toc}{subsection}{Managing Qualitative Data}

\section*{2023}\label{section-1}
\addcontentsline{toc}{section}{2023}

\markright{2023}

\subsection*{CAS Data Collection}\label{cas-data-collection}
\addcontentsline{toc}{subsection}{CAS Data Collection}

\section*{2022}\label{section-2}
\addcontentsline{toc}{section}{2022}

\markright{2022}

\subsection*{CAS Data Collection}\label{cas-data-collection-1}
\addcontentsline{toc}{subsection}{CAS Data Collection}

\section*{2021}\label{section-3}
\addcontentsline{toc}{section}{2021}

\markright{2021}

\subsection*{CAS Data Collection}\label{cas-data-collection-2}
\addcontentsline{toc}{subsection}{CAS Data Collection}

\chapter*{Presentations}\label{presentations}
\addcontentsline{toc}{chapter}{Presentations}

\markboth{Presentations}{Presentations}

\section*{2023}\label{section-4}
\addcontentsline{toc}{section}{2023}

\markright{2023}

\subsection*{Open Science on Track}\label{open-science-on-track}
\addcontentsline{toc}{subsection}{Open Science on Track}

\subsection*{DoY Network Lunch}\label{doy-network-lunch}
\addcontentsline{toc}{subsection}{DoY Network Lunch}

\section*{2022}\label{section-5}
\addcontentsline{toc}{section}{2022}

\markright{2022}

\subsection*{OS Platform}\label{os-platform}
\addcontentsline{toc}{subsection}{OS Platform}

\section*{2021}\label{section-6}
\addcontentsline{toc}{section}{2021}

\markright{2021}

\subsection*{Child Health}\label{child-health}
\addcontentsline{toc}{subsection}{Child Health}

\subsection*{OSCoffee}\label{oscoffee}
\addcontentsline{toc}{subsection}{OSCoffee}




\end{document}
